% ============================
% Configuración de colores y enlaces
% ============================

\usepackage{xcolor}
\usepackage{titlesec}
\usepackage{hyperref}
% --- Pie de página personalizado ---
\usepackage{fancyhdr}
\pagestyle{fancy}

% --- Definir colores personalizados ---
\definecolor{grisPie}{RGB}{175,175,180} % tono marrón rojizo
\definecolor{marronRojo}{RGB}{150,40,27} % tono marrón rojizo
\definecolor{azulEnlace}{RGB}{218,100,84} % azul para enlaces


% Limpia encabezado y pie actuales
\fancyhf{}
% Define el texto del pie de página centrado
\fancyfoot[C]{\textcolor{grisPie}{\small Revista Aprendiendo R y Ciencia de Datos Vol.~01}}

% (Opcional) número de página a la derecha
\fancyfoot[R]{\thepage}

% Ajusta altura del encabezado/pie si se corta
\setlength{\headheight}{14pt}
\setlength{\footskip}{20pt}




% --- Configurar color de títulos ---
\titleformat{\section}
  {\color{marronRojo}\normalfont\Large\bfseries}
  {\thesection}{1em}{}

\titleformat{\subsection}
  {\color{marronRojo}\normalfont\large\bfseries}
  {\thesubsection}{1em}{}

\titleformat{\subsubsection}
  {\color{marronRojo}\normalfont\normalsize\bfseries}
  {\thesubsubsection}{1em}{}

% Cambiando el titulo de la tabla de contenidos
\renewcommand{\contentsname}{Revista Aprendiendo R Vol. 01}

% --- Configurar enlaces en color azul ---
\hypersetup{
  colorlinks=true,
  linkcolor=azulEnlace,
  urlcolor=azulEnlace,
  citecolor=azulEnlace,
  % Metadatos
  pdfauthor={Miguel Ángel Figueroa},
  pdftitle={Manual de R Markdown},
  pdfsubject={Documentación reproducible en ciencia de datos},
  pdfkeywords={R Markdown, Ciencia de Datos, Visualización, Estadística, Reproducibilidad},
  pdfcreator={R Markdown + Pandoc},
  pdfproducer={XeLaTeX}
}
